While ontologies can be represented using a number of different formalisms, for the use in automated reasoning a trade-off must be made between expressivity and practicality. For example, first-order logic (FOL) is more expressive than propositional logic, but this added expressivity comes at the cost of decidability. In addition to decidability, scalability must also be considered in the choice and design of the used representation of the knowledge.

\emph{Description logics} (DL) are often used for building ontologies. They encompass a family of related knowledge representation languages and are often fragments of FOL with equality, as is the case with the description logics, $\mathcal{SROIQ}$ which is the main focus of this work. DLs are almost always designed to be decidable, and generally offer a favourable trade-off between expressivity and complexity of reasoning tasks. Different description logics have been developed for different applications, that feature varying levels of expressivity.

Description logics are also the basis of the \emph{Web Ontology Language} (OWL) \cite{motik2009owl_spec, hitzler2009owl_primer}, which is a World Wide Web Consortium (W3C) recommendation and is extensively used as part of the semantic web. While the OWL 2 language is based on $\mathcal{SROIQ}$, OWL 2 also defines three so-called profiles that are fragments of the full OWL 2 language that trade-off expressive power for more efficient reasoning \cite{motik2009owl_profiles}. The OWL 2 DL profile is the most expressive of the profiles, while still being decidable, and is based on $\mathcal{SROIQ}$.

The following sections will introduce the description logic $\mathcal{SROIQ}$ \cite{horrocks2006even} and the OWL 2 language. The relation between the two will also be discussed. The description is also based on the description in \cite{rudolph2011foundations}.

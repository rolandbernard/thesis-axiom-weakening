\subsection{Interpretations} \label{interpretations}

The semantics of $\SROIQ$, similar to other description logics, are defined in a model-theoretic way. Therefore, a central notion is that of the interpretation. And \emph{interpretation} $I = \langle \Delta^I, \cdot^I \rangle$ consists of a set $\Delta^I$ called the \emph{domain} of $I$, and an \emph{interpretation function} $\cdot^I$. The interpretation function maps the vocabulary elements as follows:
\begin{itemize}
  \item Each individual name $a \in N_I$ is mapped to an element $a^I \in \Delta^I$ in the domain.
  \item Each concept name $C \in N_C$ is mapped to a subset $C^I \subseteq \Delta^I$ of the domain.
  \item Each role name $r \in N_R$ is mapped to a relation $r^I \subseteq \Delta^I \times \Delta^I$ over the domain.
\end{itemize}
An interpretation maps the universal role $U$ to $U^I = \Delta^I \times \Delta^I$. We extend the interpretation function to operate over inverse roles such that  $\left(r^-\right)^I$ contains exactly those elements $\langle \delta_1, \delta_2 \rangle$ for which $\langle \delta_2, \delta_1 \rangle$ is contained in $r^I$, that is  $\left(r^-\right)^I = \{ \langle \delta_1, \delta_2 \rangle \mid \langle \delta_2, \delta_1 \rangle \in r^I \}$. Further, we define the extension of the interpretation function to complex concepts inductively as follows:
\begin{itemize}
  \item The top concept is true for every individual in the domain, therefore $\top^I = \Delta^I$.
  \item The bottom concept is true for no individual, hence $\bot^I = \emptyset$ where $\emptyset$ represents the empty set.
  \item Nominal concepts contain exactly the specified individuals, that is $\{ a \}^I = \{ a^I \}$.
  \item $\lnot C$ yields the complement of the extension of $C$, thus $(\lnot C)^I = \Delta^I \setminus C^I$.
  \item $C \sqcup D$ denotes all individuals that are in either the extension of $C$ or in that of $D$, hence $(C \sqcup D)^I = C^I \cup D^I$.
  \item $C \sqcap D$ on the other hand, denotes all elements of the domain that are in the extension of both $C$ and $D$, which can be expressed as $(C \sqcap D)^I = C^I \cap D^I$.
  \item $\exists R.C$ holds for all individuals that are connected by some element in the extension of $R$ to an individual in the extension of $C$, formally $(\exists R.C)^I = \{ \delta_1 \in \Delta^I \mid \exists \delta_2 \in \Delta^I \, . \; \langle \delta_1, \delta_2 \rangle \in R^I \, \land \, \delta_2 \in C^I  \}$.
  \item $\forall R.C$ refers to all domain elements for which all elements in the extension of $R$ connect them to elements in the extension of $C$, that is $(\forall R.C)^I = \{ \delta_1 \in \Delta^I \mid \forall \delta_2 \in \Delta^I \, . \; \langle \delta_1, \delta_2 \rangle \in R^I \to \delta_2 \in C^I \}$.
  \item $\exists R.\self$ indicates all individuals that the extension of $R$ connects to themselves, hence we let $(\exists R.\self)^I = \{ \delta \in \Delta^I \mid \langle \delta, \delta \rangle \in R^I\}$.
  \item $\less n R.C$ represents those individuals that have at most $n$ other individuals they are $R$-related to in the extension of the concept $C$, that is  $(\less n R.C)^I = \{ \delta_1 \in \Delta^I \mid \left| \{ \delta_2 \in \Delta^I \mid \langle \delta_1, \delta_2 \rangle \in R^I \, \land \; \delta_2 \in C^I \} \right| \less n \}$, where $|X|$ denotes the cardinality of a set $X$.
  \item $\more n R.C$ corollary to the case above indicates the elements of the domain that have at least $N$ such $R$-related elements, \\$(\more n R.C)^I = \{ \delta_1 \in \Delta^I \mid \left| \{ \delta_2 \in \Delta^I \mid \langle \delta_1, \delta_2 \rangle \in R^I \, \land \; \delta_2 \in C^I \} \right| \more n \}$.
\end{itemize}

\begin{example}
  Given the interpretation $I = \langle \Delta^I, \cdot^I \rangle$ where $\Delta^I = \{ 1, 2, 3, 4, 5, 6, 7 \}$ and $P^I = \{ 2, 3, 5, 7 \}$, $E^I = \{ 2, 4, 6 \}$, $o^I = 1$, and $s^I = \{ \langle 1, 2 \rangle,\allowbreak \langle 2, 3 \rangle,\allowbreak \langle 3, 4 \rangle,\allowbreak \langle 4, 5 \rangle,\allowbreak \langle 5, 6 \rangle,\allowbreak \langle 6, 7 \rangle \}$, the extended interpretation function evaluates the following example as shown.
  \begin{align*}
    \top^I &= \{ 1, 2, 3, 4, 5, 6, 7 \} &
    (\lnot P)^I &= \{ 1, 4, 6 \} \\
    (P \cup E)^I &= \{ 2, 3, 4, 5, 6, 7 \} &
    (P \cap E)^I &= \{ 2 \} \\
    (\exists s.P)^I &= \{ 1, 2, 4, 6 \} &
    (\exists s.E \cup \exists s^-.E)^I &= \{ 1, 3, 5, 7 \} \\
    (\exists s^-.\nominal o)^I &= \{ 2 \} &
    (\exists s.\nominal o)^I &= \emptyset
  \end{align*}
\end{example}

\subsection{Satisfaction of Axioms} \label{satisfaction-of-axioms}

The main purpose of this extended interpretation function is to determine the satisfaction of axioms. We define now when an axiom $\alpha$ is true, or holds, in a specific interpretation $I$. If this is the case, the interpretation $I$ satisfies $\alpha$, written $I \models \alpha$. We also say that $I$ is a model of $\alpha$.
\begin{itemize}
  \item A RIA $S_1 \circ \cdots \circ S_n \sqsubseteq R$ holds in $I$ if and only if for each sequence $\delta_1, \dots, \delta_{n + 1} \in \Delta^I$ for which $\langle \delta_i , \delta_{i + 1} \rangle \in S_i^I$ for all $i = 1, \cdots, n$, also $\langle \delta_1 , \delta_n \rangle \in R^\mathcal{i}$ is satisfied. Equivalently, we can write $I \models S_1 \circ \cdots \circ S_n \sqsubseteq R \iff S_1^I \circ \cdots \circ S_n^I \subseteq R^I$ where $\circ$ denotes the composition of relations.
  \item A role disjointness axiom $\disjoint(S, R)$ hold if and only if the extensions of $R$ and $S$ are disjoint, formally $I \models \disjoint(S, R) \iff S^I \cap R^I = \emptyset$.
  \item A GCI $C \sqsubseteq D$ is true if and only if the extension of $C$ is fully contained in the extension of $D$, hence $I \models C \sqsubseteq D \iff C^I \subseteq D^I$.
  \item A concept assertion $C(a)$ holds if and only if the individual that $a$ is mapped to by $\cdot^I$ is in the extension of the concept $C$, therefore $I \models C (a) \iff a^I \in C^I$.
  \item A role assertion $R(a, b)$ holds if and only if the individuals denoted by the names $a$ and $b$ are connected in the extension of $R$, thus $I \models R(a, b) \iff \langle a^I, b^I \rangle \in R^I$.
  \item A negative role assertion $\lnot R(a, b)$ is true exactly then when the corresponding role assertion $R(a, b)$ is false. Equivalently, $I \models \lnot R(a, b) \iff \langle a^I, b^I \rangle \not\in R^I$.
  \item An equality assertion $a = b$ holds if and only if the individuals identified by $a$ and $b$ are the same element of the domain, formally written $I \models a = b \iff a^I = b^I$.
  \item Dual to the above, $a \not = b$ holds if and only if the names $a$ and $b$ denote different elements, accordingly $I \models a \not= b \iff a^I \not= b^I$.
\end{itemize}
We say a set of axioms holds in an interpretation $I$ if and only if every axiom of the set hold in $I$. Accordingly, $I$ satisfies a ontology $\Omc$, written $I \models \Omc$, if and only if $I$ satisfies every axiom $\alpha \in \Omc$ of the ontology, i.e., $I \models \Omc \iff \forall \alpha \in \Omc : I \models \alpha$. If $I$ satisfies $\Omc$, we say $I$ is a model of $\Omc$.

\begin{example}
  Given the interpretation $I = \langle \Delta^I, \cdot^I \rangle$ where $\Delta^I = \{ 1, 2, 3, 4, 5, 6, 7 \}$ and $P^I = \{ 2, 3, 5, 7 \}$, $E^I = \{ 2, 4, 6 \}$, $o^I = 1$, $t^I = 2$, and $s^I = \{ \langle 1, 2 \rangle,\allowbreak \langle 2, 3 \rangle,\allowbreak \langle 3, 4 \rangle,\allowbreak \langle 4, 5 \rangle,\allowbreak \langle 5, 6 \rangle,\allowbreak \langle 6, 7 \rangle \}$, the following axioms are satisfied by $I$.
  \begin{align*}
    &P \sqsubseteq \top &
    &P \sqcap \lnot (\exists s^-.\nominal o) \sqsubseteq \lnot E \\
    &P \sqcup E \sqsubseteq \lnot \nominal o &
    &\disjoint(s, s^-) \\
    &s(o, t) &
    &o \not= t
  \end{align*}
  The following axioms on the other hand are not satisfied by $I$.
  \begin{align*}
    &P \sqsubseteq \lnot E &
    &\top \sqsubseteq P \sqcup E &
    &\top \sqsubseteq \exists s.\top \\
    &s \circ s \sqsubseteq s &
    &s \sqsubseteq s^- &
    &\top \sqsubseteq{} \more 1 s.E \\
    &\lnot s(o, t) &
    &o = t &
    &s(t, o)
  \end{align*}
\end{example}

\subsection{Reasoning tasks} \label{reasoning-tasks}

In general, logic-based knowledge representation is useful for the ability to perform reasoning tasks on ontologies. Several reasoning tasks can be performed by a reasoner in DLs. In this section, we will have a look at three basic reasoning tasks, and how they can be reduced to each other. While there exist other reasoning tasks, this section will focus on \emph{ontology consistency}, \emph{axiom entailment}, and \emph{concept satisfiability}.

\subsubsection{Ontology consistency} \label{knowledge-base-satisfiability}

An ontology $\Omc$ is consistent if and only if there exists a model $I \models \Omc$ for $\Omc$. Otherwise, the ontology is called inconsistent. As discussed in \cref{ontology-bugs}, an inconsistent ontology can be a sign of modelling errors. An inconsistent ontology entailed every statement, and as such all information extracted from it is useless. Therefore, an inconsistent ontology is generally undesirable. Furthermore, both the task of deciding concept satisfiability and axiom entailment can be reduced to deciding ontology consistency.

\subsubsection{Axiom entailment} \label{axiom-entailment}

An axiom $\alpha$ is entailed by $\Omc$ if every model $I \models \Omc$ of the ontology also satisfies the axiom $I \models \alpha$. We also write this as $\Omc \models \alpha$ and say that $\alpha$ is a consequence of $\Omc$. Deciding axiom entailment is an important task in order to derive new information from the collected knowledge. If $\alpha$ is entailed by the empty ontology, $\alpha$ is said to be a \emph{tautology}. Further, the \emph{set of consequences} of $\Omc$ if the set of all axioms which are entailed by $\Omc$, we write $\op{Con}(\Omc) = \{ \alpha \mid \Omc \models \alpha \}.$ It is clear that the set of consequences will always be infinite since there is an infinite number of tautologies.

The problem of axiom entailment can be reduced to determining the satisfiability of a modified ontology. This is achieved by using an axiom $\beta$ that imposes the opposite restriction to $\alpha$, to be more precise, for all interpretations $I \models \alpha \iff I \not\models \beta$. If $\alpha$ is entailed by $\Omc$, it must hold in every model of $\Omc$, hence $\beta$ must not hold in any model. It follows that the extended ontology $\Omc \cup \{ \, \beta \, \}$ has no new model, and is therefore unsatisfiable. We can consequently solve the axiom entailment problem by testing for the satisfiability of a modified ontology if we can find such an opposing axiom for $\alpha$. For some cases in $\SROIQ$ finding such an opposite is obvious, for others the desired behaviour must be emulated with a set of axioms. \Cref{tab:entailment-reduction} shows the correspondence for every type of $\SROIQ$ axiom.
\begin{table}[ht]
  \centering
  \begin{tabular}{|l|l|}
    \hline
    $\alpha$ & $B$ \\
    \hline
    $S_1 \circ \cdots \circ S_n \sqsubseteq R$ & $S_1(a_1, a_2)$, $\dots$, $S_n(a_n, a_{n + 1})$, and $\lnot R(a_1, a_{n + 1})$ \\
    $\disjoint(S, R)$ & $S(a, b)$ and $R(a, b)$ \\
    $C \sqsubseteq D$ & $(C \sqcap \lnot D)(a)$ \\
    $C(a)$ & $\lnot C(a)$ \\
    $R(a, b)$ & $\lnot R(a, b)$ \\
    $\lnot R(a, b)$ & $R(a, b)$ \\
    $a = b$ & $a \not= b$ \\
    $a \not= b$ & $a = b$ \\
    \hline
  \end{tabular}
  \label{tab:entailment-reduction}
  \caption[Axioms substitution for reducing entailment to satisfiability]{The axioms in $B$ together have the “opposite” meaning of those in $\alpha$. $a$, $a_i$, and $b$ are assumed to not appear in $\Omc$. This means, checking the entailment of $\alpha$ with respect to $\Omc$, is equivalent to checking the unsatisfiability of $\Omc \cup B$.}
\end{table}

\subsubsection{Concept satisfiability} \label{concept-satisfiability}

A concept $C$ is satisfiable with respect to $\Omc$ if and only if there exists a model of the ontology $I \models \Omc$ such that the extension of $C$ is not empty, i.e., $C^I \not= \emptyset$. A concept which is not satisfiable is called unsatisfiable. Clearly, some concepts are unsatisfiable with respect to every ontology, for example, $\bot$ or $A \sqcap \lnot A$. However, similar to an unsatisfiable ontology, an unsatisfiable atomic concept may be an indication of a modelling mistake.

Like axiom entailment, concept satisfiability can be reduced to ontology satisfiability. If a concept is unsatisfiable, every model $I \models \Omc$ maps the concept to the empty set, that is $C^I = \emptyset$. Since the other direction is trivial, we can rewrite this as $C^I \subseteq \emptyset$. It follows that since $\bot^I = \emptyset$ every such model satisfies $I \models C \sqsubseteq \bot$, meaning $\Omc \models C \sqsubseteq \bot$. We conclude that we can test for the unsatisfiability of a concept $C$ by checking for entailment of $C \sqsubseteq \bot$.

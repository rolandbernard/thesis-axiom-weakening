
The evaluation of the proposed refinement and axiom weakening operators has been carried out using the implementation described in \cref{prototype} using different ontologies from BioPortal \cite{whetzel2011bioportal,bioportal}. Additionally, the pizza ontology \cite{pizzaontology} was included in the testing. The chosen ontologies are of different sizes and use a varying amount of expressive features. Some characteristics of the used ontologies are shown in \cref{table:ontologies}. On average, they contain about 289 axioms, 73 concept names, 29 role names, and 168 subconcepts. Some additional evaluation results that did not fit into this chapter can be found in \cref{eval-appendix}.

\begin{table}[ht]
  \scriptsize
  \centering
  \addtolength{\tabcolsep}{-0.75mm}
  \begin{tabular}{|l|llrrrr|}
    \hline
    Abbreviation & Name & Expressivity & Axioms & Concepts & Roles & Subcon. \\
    \hline
    admin & Nurse Administrator & $\mathcal{ALCHOIF}$ & 229 & 42 & 29 & 144 \\
    ahso & Animal Health Surveillance Ontology & $\mathcal{ALCRIF}$ & 166 & 38 & 31 & 104 \\
    cdao & Comparative Data Analysis Ontology & $\mathcal{ALCROIQ}$ & 437 & 132 & 68 & 375 \\
    cdpeo & Chronic Disease Patient Education & $\mathcal{ALCHF}$ & 422 & 41 & 31 & 170 \\
    covid19-ibo & Covid-19 Impact on Banking Ontology & $\mathcal{ALCH}$ & 288 & 160 & 33 & 227 \\
    ecp & Electronic Care Plan & $\mathcal{ALCRQ}$ & 127 & 33 & 17 & 99 \\
    emo & Enzyme Mechanism Ontology & $\mathcal{ALCHQ}$ & 368 & 157 & 24 & 255 \\
    evi & Evidence Graph Ontology & $\mathcal{ALCRI}$ & 143 & 30 & 38 & 69 \\
    falls & Falls Prevention & $\mathcal{ALCH}$ & 79 & 30 & 20 & 35 \\
    fo & Fern Ontology & $\mathcal{ALCHI}$ & 59 & 31 & 4 & 46 \\
    gbm & Glioblastoma & $\mathcal{ALCIF}$ & 603 & 108 & 28 & 227 \\
    gfvo & Genomic Feature and Variation Ontology & $\mathcal{ALCH}$ & 332 & 102 & 30 & 170 \\
    koro & Knowledge Object Reference Ontology & $\mathcal{ALCHI}$ & 262 & 110 & 19 & 194 \\
    lico & Liver Case Ontology & $\mathcal{ALCHQ}$ & 366 & 93 & 36 & 230 \\
    mamo & Mathematical Modelling Ontology & $\mathcal{ALCR}$ & 229 & 107 & 3 & 154 \\
    mpio & Minimum PDDI Information Ontology & $\mathcal{ALCH}$ & 38 & 30 & 14 & 45 \\
    pizza & Pizza Ontology & $\mathcal{SHOIN}$ & 1131 & 100 & 8 & 376 \\
    provo & Provenance Ontology & $\mathcal{ALCRIN}$ & 170 & 31 & 42 & 128 \\
    qudt & Quantities, Units, Dimensions, and Types & $\mathcal{SHOIQ}$ & 293 & 74 & 73 & 177 \\
    trans & Nurse Transitional & $\mathcal{ALCROIF}$ & 244 & 44 & 22 & 123 \\
    triage & Nurse triage & $\mathcal{ALCHF}$ & 132 & 33 & 29 & 129 \\
    vio & Vaccine Investigation Ontology & $\mathcal{ALCRI}$ & 249 & 81 & 44 & 235 \\
    \hline
  \end{tabular}
  \caption{The ontologies used for evaluation. The number of axioms, concept names, role names, and subconcepts are taken after applying preprocessing.}
  \label{table:ontologies}
\end{table}

Since the ontologies use OWL 2, the axioms and concepts do not map directly to \SROIQ. It is possible to directly apply axiom weakening to OWL 2 DL ontologies, a description of such a scheme and why it might be beneficial can be found in \cref{weakening-owl-2-dl}. However, to follow the definitions laid out in \cref{weakening-sroiq}, the OWL 2 DL axioms have been translated to \SROIQ axioms. A detailed description of the mapping between OWL 2 DL and \SROIQ used for this evaluation can be found in \cref{owl-to-sroiq}. While some axioms have a direct translation, others have been replaced by a set of axioms that together are equivalent. During the preprocessing, we further removed annotation axioms and axioms related to data properties and any axiom that caused a violation of the OWL 2 DL profile, as our weakening operator does not handle them.

Claims about runtime should generally only be used for relative comparisons and rough estimates. While they will obviously vary greatly based on hardware choices, all evaluations here have been performed on an Intel Coffee Lake system running at around 4GHz for the duration of the experiment. Unless otherwise indicated, the FaCT++ reasoner \cite{tsarkov2006fact++,factpp} was used for evaluation.

\section{The Quality of Repairs}\label{eval-quality}


% TODO

\begin{algorithm}[ht]
  \begin{algorithmic}
    \While{$\Omc$ is inconsistent}
    \State $\phi_\textnormal{bad} \gets \textsc{FindBadAxiom}(\Omc)$
    \State $\Omc \gets \Omc \setminus \{\phi_\textnormal{bad}\}$
    \EndWhile
    \State Return $\Omc$
  \end{algorithmic}
  \caption{\textsc{RepairOntologyRemove}($\Omc$)}
  \label{algo:repair-remove}
\end{algorithm}


\section{Caching in Cover Computations}


As discussed in \cref{cache-impl}, to accelerate the computation of upward and downward covers, a cache was added to avoid repeated calls to the reasoner when the necessary information can already be inferred from previous computations. This section will discuss the experiments performed to evaluate the effectiveness of this cache.

Three versions of the upward and downward cover have been considered, the uncached version that calls the reasoner for every subsumption query, a version that caches only the exact queries that were already made previously, and finally the version that additionally infers some additional information from the transitivity of subsumption as listed in \cref{algo:cached-subs}. The experiment uses some of the same ontologies as have been used for the previously discussed evaluation of the repair quality. From the logical axioms of each ontology were selected, uniformly at random, one hundred groups with a fixed number of axioms. The same axioms may be selected multiple times. Each of the groups was then tested with a separate instance of the cache. The rationale behind testing with different group sizes is that the cache will obviously have a greater impact when the same cache can be reused for more cover computations. In each test run, the axiom weakening operator was applied to each axiom and the number of reasoner calls and (real) time taken were measured. The weakening operator used the complete (after preprocessing) ontology as the reference ontology. The test has further been performed using different OWL 2 DL reasoners. The final results of the evaluation can be seen in \cref{table:results-cache-calls} and are visualized in \cref{fig:results-cache-calls} and \cref{fig:results-cache-time}. Note that a logarithmic y-axis has been used for both plots.

\begin{table}[ht]
  \scriptsize
  \centering
  \begin{tabular}{|l|rrr|rrr|rrr|}
    \cline{2-10}
    \multicolumn{1}{l|}{} & \multicolumn{9}{c|}{\hspace{-4mm}Reasoner calls per weakening} \\
    \multicolumn{1}{l|}{} & \multicolumn{3}{c}{Full caching} & \multicolumn{3}{c}{Simple caching} & \multicolumn{3}{c|}{No caching} \\
    \multicolumn{1}{l|}{} & \multicolumn{1}{c}{1} & \multicolumn{1}{c}{10} & \multicolumn{1}{c}{100} & \multicolumn{1}{c}{1} & \multicolumn{1}{c}{10} & \multicolumn{1}{c}{100} & \multicolumn{1}{c}{1} & \multicolumn{1}{c}{10} & 100 \\
    \hline
    admin & 1096 & 222 & 35
      & 15138 & 2115 & 236
      & 41105 & 41288 & 41751 \\
    cdpeo & 2110 & 522 & 75
      & 13621 & 2694 & 312
      & 29051 & 28617 & 29315 \\
    emo & 2652 & 1355 & 258
      & 7781 & 2784 & 619
      & 12524 & 12134 & 12552 \\
    gbm & 3019 & 1074 & 168
      & 12572 & 3284 & 503
      & 19490 & 21330 & 20801 \\
    gfvo & 1737 & 575 & 80
      & 2828 & 1524 & 293
      & 4058 & 4104 & 4003 \\
    koro & 2006 & 591 & 80
      & 5154 & 2272 & 382
      & 7271 & 7181 & 7422 \\
    mamo &  1984 & 511 & 61
      & 3557 & 1488 & 234
      & 5060 & 5059 & 4998 \\
    \hline
    Overall & 2086 & 693 & 108
      & 8664 & 2309 & 368
      & 16937 & 17102 & 17263 \\
    \hline
  \end{tabular}
  \caption{Results of the evaluation of cache effectiveness. The mean number of reasoner calls required for a single application of the axiom weakening operator on randomly selected axioms of the ontology is given for different degrees of cache reuse. The cache is reused of one, ten, or one hundred successive operator applications.}
  \label{table:results-cache-calls}
\end{table}

\begin{figure}[ht]
  \centering
  \includegraphics[width=\textwidth]{resources/calls-cache-bar.png}
  \caption{The mean number of reasoner calls needed for a single application of the axiom weakening operator, averaged over the tested ontologies.}
  \label{fig:results-cache-calls}
\end{figure}

\begin{figure}[ht]
  \centering
  \includegraphics[width=\textwidth]{resources/time-cache-bar.png}
  \caption{Average time required per application of the axiom weakening operator with different caching strategies. The results are averaged over the tested ontologies and the reasoners FaCT++, JFact, Openllet, and HermiT.}
  \label{fig:results-cache-time}
\end{figure}

We can see clearly from the results that the cache is indeed effective at lowering both the number of reasoner calls and execution time. The simple caching method alone provides a dramatic decrease in the number of reasoner calls, especially at higher group sizes. The algorithm that can additionally exploit the transitivity of the relation performs even better, also with a smaller number of weakening steps. The difference when looking at the execution time is significantly smaller. This can likely be attributed largely to internal caching performed by the reasoners. This would also explain the observed variation when it comes to the relative improvement between different reasoners. It can be concluded that the addition of the cache greatly benefits the computation of the axiom weakening operator, especially if it can be reused for many applications of the same weakening operator.



\section{Reasoner Calls and Execution Time}


The performance of the axiom weakening based repair algorithm shown in \cref{algo:repair-weaken} has also been evaluated. During each repair via axiom weakening, the required (real) time and the number of calls to the reasoner have been registered. Additionally, the number of steps taken by the repair using axiom weakening has been observed. Also, as mentioned in \cref{eval-quality}, the repair program was put under a timeout to prevent cases where the reasoning becomes unreasonably slow. The generation of inconsistent ontologies used a similar timeout, and the same procedure was used for situations in which the reasoner ran out of memory. The timeouts of the weakening procedure are shown separately from those latter cases in the results. The frequency of these cases is indicated as a percentage of the overall runs finished vs started. The results of this evaluation are presented in \cref{table:results-perf} and \cref{fig:results-perf-time}.

\begin{table}[ht]
  \scriptsize
  \centering
  \begin{tabular}{|l|r@{ }lr@{ }lr@{ }lr@{ }r|}
    \cline{2-9}
    \multicolumn{1}{l|}{} & \multicolumn{2}{c}{Steps} & \multicolumn{2}{c}{Calls} & \multicolumn{2}{c}{Time [ms]} & \multicolumn{2}{c|}{Failed} \\
    \hline
    admin & 6.3 & [4.0, 9.2] & 7621 & [5601, 10118] & 9638 & [5433, 14909] & 2\% & (2\%) \\
    ahso & 2.1 & [1.7, 2.5] & 4648 & [4228, 5131] & 11469 & [8122, 15336] & 11\% & (26\%) \\
    cdao & 2.1 & [1.8, 2.5] & 16137 & [14576, 17740] & 20767 & [15204, 27269] & 19\% & (48\%) \\
    cdpeo & 1.5 & [1.3, 1.7] & 4476 & [4250, 4716] & 2050 & [1942, 2169] & 0\% & (0\%) \\
    covid19-ibo & 1.3 & [1.2, 1.5] & 14822 & [14321, 15320] & 2208 & [2126, 2296] & 4\% & (13\%) \\
    ecp & 1.3 & [1.2, 1.5] & 2020 & [1916, 2133] & 4453 & [2738, 6939] & 1\% & (14\%) \\
    emo & 1.4 & [1.3, 1.6] & 18549 & [17473, 19615] & 6582 & [4439, 9688] & 1\% & (4\%) \\
    evi & 5.0 & [3.9, 6.4] & 5955 & [4680, 7425] & 4719 & [3408, 6349] & 19\% & (64\%) \\
    falls & 1.5 & [1.3, 1.6] & 1242 & [1161, 1331] & 874 & [843, 909] & 1\% & (5\%) \\
    fo & 1.6 & [1.4, 1.8] & 1295 & [1179, 1423] & 1090 & [1025, 1164] & 38\% & (61\%) \\
    gbm & 1.5 & [1.3, 1.6] & 7608 & [7070, 8131] & 2954 & [2808, 3117] & 0\% & (0\%) \\
    gfvo & 1.8 & [1.6, 2.0] & 4167 & [3864, 4501] & 2404 & [2258, 2569] & 0\% & (0\%) \\
    koro & 1.9 & [1.7, 2.2] & 6195 & [5697, 6706] & 2456 & [2277, 2648] & 0\% & (0\%) \\
    lico & 2.6 & [2.2, 2.9] & 6638 & [6105, 7186] & 3709 & [3400, 4077] & 1\% & (22\%) \\
    mamo & 2.5 & [2.2, 2.9] & 4667 & [4228, 5128] & 2215 & [2046, 2403] & 0\% & (0\%) \\
    mpio & 2.2 & [1.8, 2.6] & 1908 & [1720, 2133] & 987 & [913, 1078] & 9\% & (30\%) \\
    pizza & 2.0 & [1.7, 2.3] & 7550 & [6723, 8419] & 26767 & [20554, 34014] & 28\% & (55\%) \\
    provo & 4.9 & [3.7, 6.3] & 6851 & [5383, 8465] & 8878 & [5962, 12348] & 4\% & (8\%) \\
    qudt & 1.1 & [1.0, 1.2] & 7044 & [6816, 7291] & 6658 & [4229, 9672] & 9\% & (38\%) \\
    trans & 3.2 & [1.8, 5.1] & 4384 & [3213, 5894] & 3684 & [1610, 6513] & 0\% & (5\%) \\
    triage & 3.2 & [2.4, 4.3] & 5166 & [4255, 6351] & 8979 & [6084, 13102] & 28\% & (51\%) \\
    vio & 2.5 & [2.1, 3.0] & 9476 & [8628, 10327] & 15199 & [10496, 20803] & 3\% & (7\%) \\
    \hline
  \end{tabular}
  \caption{Results of the evaluation with respect to performance. The number of weakening iterations, reasoner calls, and total repair time are given as sample mean, with the 95\% confidence interval in brackets. The frequency of failed runs is shown as percentage of repairs by weakening that were started but not completed. In parentheses, the percentage of total failed runs, including those with timeout during generation of the inconsistent ontology. Attempts that failed were not considered for the other values.}
  \label{table:results-perf}
\end{table}

\begin{figure}[ht]
  \centering
  \includegraphics[width=\textwidth]{resources/time-ontology-violin.png}
  \caption{Distribution of (real) execution time required for repairing a single ontology using axiom weakening. The two darkest blue boxes represent (together) half of the samples, with the line in the middle indicating the median. Each lighter box represents half the samples of the boxes one step darker, and all remaining outliers are marked. Attempts that failed by a timeout or other errors were not considered.}
  \label{fig:results-perf-time}
\end{figure}

As is visible from the results, the number of reasoner calls and the execution time can vary significantly. The execution times were generally reasonable when a run was able to complete within the time limit, with most of them completing within 2 minutes, even though the time limit was set to 5 minutes. A significant number of runs, however, were affected by the timeout or other errors. It has not been looked deeper into what causes these issues.



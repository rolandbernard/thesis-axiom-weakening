
Formally, an ontology is a set of statements expressed in a suitable logical language and with the purpose of describing a specific domain of interest. While ontologies can be represented using a number of different formalisms, for the use in automated reasoning a trade-off must be made between expressivity and practicality. For example, first-order logic (FOL) is more expressive than propositional logic, but this added expressivity comes at the cost of decidability. In addition to decidability, scalability must also be considered in the choice and design of the used representation of the knowledge.

\emph{Description logics} (DL) are often used for building ontologies. They encompass a family of related knowledge representation languages and are often fragments of FOL\footnote{While description logics are fragments of FOL in the sense that for every knowledge base in a given description logic, there exists a FOL theory that has the same models, the syntax used by description logics is different from the syntax used in FOL, and as such a DL axiom is not a valid FOL sentence.} with equality, as is the case with the description logics $\mathcal{SROIQ}$ which is the main focus of this work. DLs are almost always designed to be decidable, and generally offer a favourable trade-off between expressivity and complexity of reasoning tasks. Different description logics have been developed for different applications, that feature varying levels of expressivity.

In this section will briefly introduce the description logic $\mathcal{SROIQ}$ \cite{horrocks2006even, rudolph2011foundations, baader_horrocks_lutz_sattler_2017}. A more detailed description, containing more information about the semantics, can be found in \cref{sroiq-appendix}.

The syntax of $\mathcal{SROIQ}$ is based on a vocabulary of three disjoint sets $N_C$, $N_R$, $N_I$ of, respectively, \emph{concept names}, \emph{role names}, and \emph{individual names}. The sets of, respectively, $\mathcal{SROIQ}$ \emph{roles} and $\mathcal{SROIQ}$ \emph{concepts} are generated by the following grammar.
\begin{alignat*}{2}
  R, S &::={} && U \mid r \mid r^{-} \enspace,\\
  C &::= && \bot \mid \top \mid A \mid \neg C \mid C \sqcap C \mid C \sqcup C \mid \forall R.C \mid \exists R.C \mid \\ 
  &&& \geq n~S.C \mid \leq n~S.C \mid \exists S.\mathit{self} \mid \{ i \} \enspace,
\end{alignat*}
\noindent where $A \in N_C$ is a concept name, $r \in N_R$ is a role name, $i \in N_I$ is an individual name and $n \in \mathbb{N}_0$ is a non-negative integer. 
%
$U$ is the universal role. $S$ is a \emph{simple role} in the RBox $\mathcal{R}$ (see below). In the following, $\mathcal{L}(N_C, N_R, N_I)$ and $\mathcal{L}(N_R) = N_R \cup \{U\} \cup \{r^- \mid r \in N_R\}$ denote, respectively, the set of concepts and roles that can be built over $N_C$, $N_R$, and $N_I$ in $\mathcal{SROIQ}$.

We next define the notions of TBox, ABox, and (regular) RBox, of complex role inclusions, and of (non-)simple roles:
% describe the TBox, ABox and RBox statements
A \emph{TBox} $\mathcal{T}$ is a finite set of concept inclusions (GCIs) of the form $C \sqsubseteq D$ where $C$ and $D$ are concepts. The TBox is used to store terminological knowledge concerning the relationship between concepts. 
%
An \emph{ABox} $\mathcal{A}$ is a finite set of statements of the form $C(a)$, $R(a,b)$, $\lnot R (a,b)$, $a = b$, and $a \not= b$, where $C$ is a concept, $R$ is a role and $a$ and $b$ are individual names. The ABox expresses knowledge regarding individuals in the domain. 
%
An \emph{RBox} $\mathcal{R}$ is a finite set of role inclusions (RIAs) of the form $R_1 \circ \cdots \circ R_n \sqsubseteq R$, and disjoint role axioms $\mathit{disjoint}(S_1, S_2)$ where $R$, $R_1$, $\dots$, $R_n$, $S_1$, and $S_2$ are roles. $S_1$ and $S_2$ are simple (defined next) in the RBox $\mathcal{R}$. The special case of $n = 1$ is a simple role inclusion, while we call the cases where $n > 1$ \emph{complex role inclusions}. The RBox represents knowledge about the relationships between roles.

% describe simple and complex roles since it is relevant
The set of \emph{non-simple} roles in $\mathcal{R}$ is the smallest set such that: $U$ is non-simple; any role $R$ that appears as the super role of a complex RIA $R_1 \circ \cdots \circ R_n \sqsubseteq R$ where $n > 1$ is non-simple; any role $R$ that appears on the right-hand side of a simple RIA $S \sqsubseteq R$ where $S$ is non-simple, is also non-simple; and a role $r$ is non-simple if and only if $r^-$ is non-simple.
All other roles are \emph{simple}.

% describe regularity since it is relevant
For convenience, let us define the function $\mathit{inv}(R)$ such that $\mathit{inv}(r) = r^-$ and $\mathit{inv}(r^-) = r$ for all role names $r \in N_R$. 
An RBox $\mathcal{R}$ is  \emph{regular} if there exists a preorder $\preceq$, i.e., a transitive and reflexive relation, over the set of roles appearing in $\mathcal{R}$, such that $R \preceq S \iff \mathit{inv}(R) \preceq S$, and all RIAs in $\mathcal{R}$ are of the forms:
$\mathit{inv}(R) \sqsubseteq R$,
$R \circ R \sqsubseteq R$,
$S \sqsubseteq R$, $R \circ S_1 \circ \cdots \circ S_n \sqsubseteq R$,
$S_1 \circ \cdots \circ S_n \circ R \sqsubseteq R$, or
$S_1 \circ \cdots \circ S_n \sqsubseteq R$,
where $n > 1$ and $R$, $S$, $S_1, \cdots, S_n$ are roles such that $S \preceq R$, $S_i \preceq R$, and $R \not\preceq S_i$ for $i = 1, \dots, n$. The definitions for simple roles and regularity used here align with the global restrictions in OWL 2 DL \cite{motik2012ontology}.

\begin{definition}
    A $\mathcal{SROIQ}$  ontology $\mathcal{O} = \mathcal{T} \cup \mathcal{A} \cup \mathcal{R}$ consists of a TBox $\mathcal{T}$, an ABox $\mathcal{A}$, and a RBox $\mathcal{R}$ in the language of $\mathcal{SROIQ}$, and where the RBox $\mathcal{R}$ is regular.
\end{definition}

% describe the semantics
% whats a interpretation, when is it a model
The semantics of $\mathcal{SROIQ}$ is  defined using \emph{interpretations} $I = \langle \Delta^I, \cdot^I \rangle$, where $\Delta^I$ is a non-empty \emph{domain} and $\cdot^I$ is a function associating to each individual name $a$ an element of the domain $a^I \in \Delta^I$, to each concept $C$ a subset of the domain $C^I \subseteq \Delta^I$, and to each role $R$ a binary relation on the domain $R^I \subseteq \Delta^I \times \Delta^I$; see \cite{baader_horrocks_lutz_sattler_2017,HorrocksKutzSattlerKR2006} or \cref{sroiq-appendix} for further details. An interpretation $I$ is a \emph{model} for $\mathcal{O}$ if it satisfies all the axioms in $\mathcal{O}$.

% describe subsumptions of concepts and roles
Given two concepts $C$ and $D$ we say that $C$ is \emph{subsumed} by $D$ (or $D$ \emph{subsumes} $C$) with respect to the ontology $\mathcal{O}$, written $C \sqsubseteq_\mathcal{O} D$, if $C^I \subseteq D^I$ in every model $I$ of $\mathcal{O}$. Further, $C$ is \emph{strictly subsumed by} $D$, written $C \sqsubset_\mathcal{O} D$, if $C \sqsubseteq_\mathcal{O} D$ but not $D \sqsubseteq_\mathcal{O} C$. Analogously, given two roles $R$ and $S$, $R$ is subsumed by $S$ with respect to $\mathcal{O}$ ($R \sqsubseteq_\mathcal{O} S$) if $R^I \sqsubseteq S^I$ in all models $I$ of $\mathcal{O}$. Again, $R \sqsubset_\mathcal{O} S$ holds if $R \sqsubseteq_\mathcal{O} S$ but not $D \sqsubseteq_\mathcal{O} C$.


% Explain what the Web Ontology Language (OWL) is
Description logics are also the basis of the \emph{Web Ontology Language} (OWL) \cite{hitzler2012primer,motik2012ontology}, which is a World Wide Web Consortium (W3C) recommendation and is extensively used as part of the semantic web. While the OWL 2 DL language is based on $\SROIQ$, OWL 2 also defines three so-called profiles that are fragments of the full OWL 2 language that trade-off expressive power for more efficient reasoning \cite{motik2012profiles,motik2012ontology}. OWL 2 EL is a subset based on the $\mathcal{EL}^{++}$ description logic, and allows the basic reasoning task to be performed in polynomial time. This fragment is useful for ontologies with many classes and properties. OWL 2 QL is designed for applications with many instances and where query answering is an important task. OWL 2 RL is designed such that reasoning tasks can be implemented using a rule-based reasoning engine. The OWL 2 DL profile, which is the most expressive of the profiles that is still decidable\footnote{The most expressive profile, OWL 2 Full, is not decidable.}, is based on $\SROIQ$. This thesis therefore focuses on OWL 2 DL.

It is important to note that while OWL 2 is based on description logics like \SROIQ, it provides many more axiom and concept expession types than are natively available in \SROIQ. It provides for example axioms that allow specifying disjointness between concepts, or ones that allow the description of equivalent roles. These additional axioms do, however, not make the language more expressive, as all of them can equivalently be expressed using one or more \SROIQ axioms. In contrast, OWL 2 DL also provides some features that can not be reproduced in pure \SROIQ, e.g., annotations, datatypes and data properties. For the rest of this thesis, these will not be considered and they have been removed from the ontolgies used in the evaluation, as will be expalined in further detail in \cref{evaluation}.

In addition to the different profiles, there exist also different syntax for OWL 2. The syntax defined and used by the main specification \cite{motik2012ontology} is the so-called functional syntax. An alternative ontology format supporting OWL 2 is the XML/RDF based format \cite{beckett2004rdf,motik2009rdf}, which is the main format used for exchanging ontologies between different tools. It is the only format whose support is mandated for compliant tools by the specification. The XML version of RDF format is not designed to be presented and manipulated directly by human users, but is suitable for machine processing and interoperability. The RDF format can be used also for other forms of data and can be written also in an alternative non-XML syntax that is more human-readable called Turtle \cite{beckett2008turtle}. Another popular format for OWL 2 is the Manchester syntax \cite{horridge2009manchester} which is also the default used in the popular ontology development tool Protégé (see further).

Highly optimized reasoners exist for different OWL profiles. reasoners are programs that, given an ontology, can perform certain reasoning tasks like checking for consistency of the ontology and testing whether axioms are entailed or not. Some reasoners specialize on the more restricted profiles, e.g., ELK \cite{kazakov2014elk} supports only the OWL 2 EL profile. Others, like the one used for this thesis have complete or nearly complete support for OWL 2 DL, and therefore also \SROIQ. For this thesis, as will be explained in more detail in \cref{implementation}, we are using the reasoners by means of the OWL API \cite{horridge2011owl}. The OWL API is a Java library that provides data structures and utilities to represent OWL 2 ontologies, load and store them to disk, and perform a number of basic transformations. Further, it provides a common way of interfacing with different reasoners to query consistency, entailment, and class hierarchies.

Protégé \cite{musen2015protege} is a popular tool for ontology development build on top of the OWL API. It allows reading, writing, viewing, and modifying ontologies in all main formats supporting OWL 2. As mentioned above, it used the Manchester syntax for displaying ontology axioms and has support for plugins. These can be used to add different functionalities like reasoners, debuggers, or ontology analysis tools. In \cref{protege}, we show the implementation of a Protégé plugin supporting the application of the algorithms proposed in this thesis to the ontologies loaded in the editor.


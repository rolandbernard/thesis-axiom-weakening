% Introduce what description logcs are 
Formally, an ontology is a set of statements expressed in an appropriate logical language and with the purpose of describing a specific domain of interest. While ontologies can be represented using a number of different formalisms, for the use in automated reasoning a trade-off must be made between expressivity and practicality. For example, \emph{first-order logic} (FOL) is more expressive than propositional logic, but this added expressivity comes at the cost of decidability. In addition to decidability, scalability must also be considered in the choice and design of the used representation of the knowledge.

\emph{Description logics} (DL) are often used for building ontologies. They include a family of related knowledge representation languages and are often fragments of FOL\footnote{While DLs are fragments of FOL in the sense that for every ontology in a given DL, there exists a FOL theory that has the same models, the syntax used by DLs is different from the syntax used in FOL, and as such a DL axiom is not a valid FOL sentence.} with equality, as is the case with the DL $\SROIQ$ which is the main focus of this work. DLs are almost always designed to be decidable, and generally offer a favourable trade-off between expressivity and complexity of reasoning tasks. Different DLs have been developed for different applications, and allow for varying levels of expressivity.

This section will briefly introduce the DL $\SROIQ$ \cite{horrocks2006even, rudolph2011foundations, baader_horrocks_lutz_sattler_2017}. A more detailed description, including more explanations and examples, can be found in \cref{sroiq-appendix}.

\subsection{The Syntax of \SROIQ} \label{sroiq-syntax}

% describe the SROIQ syntax
% describe roles and concepts
The syntax of \SROIQ is based on a \emph{vocabulary} of three finite disjoint sets $N_C$, $N_R$, $N_I$ of, respectively, \emph{concept names}, \emph{role names}, and \emph{individual names}. The sets of, respectively, \SROIQ  \emph{roles} and \SROIQ \emph{concepts} are generated by the following grammar.
\begin{alignat*}{2}
  R, S &::={} && U \mid r \mid r^{-} \enspace,\\
  C &::= && \bot \mid \top \mid A \mid \neg C \mid C \sqcap C \mid C \sqcup C \mid \forall R.C \mid \exists R.C \mid \\ 
  &&& \more n S.C \mid \less n S.C \mid \exists S.\self \mid \nominal a \enspace,
\end{alignat*}
where $A \in N_C$ is a concept name, $r \in N_R$ is a role name, $a \in N_I$ is an individual name and $n \in \mathbb{N}_0$ is a non-negative integer. 
%
$U$ is the universal role. $S$ is a \emph{simple role} in the RBox $\Rmc$ (see below). In the following, $\Lmc(N_C, N_R, N_I)$ and $\Lmc(N_R) = N_R \cup \{U\} \cup \{r^- \mid r \in N_R\}$ denote, respectively, the set of concepts and roles that can be built over $N_C$, $N_R$, and $N_I$ in \SROIQ.

We next define the notions of TBox, ABox, and (regular) RBox, of complex role inclusions, and of (non-)simple roles:
% describe the TBox, ABox and RBox statements
A \emph{TBox} $\Tmc$ is a finite set of \emph{concept inclusions} (GCIs) of the form $C \sqsubseteq D$ where $C$ and $D$ are concepts. The TBox is used to store terminological knowledge concerning the relationship between concepts. 
%
An \emph{ABox} $\Amc$ is a finite set of statements of the form $C(a)$, $R(a,b)$, $\lnot R (a,b)$, $a = b$, and $a \not= b$, where $C$ is a concept, $R$ is a role and $a$ and $b$ are individual names. The ABox expresses knowledge regarding individuals in the domain. 
%
An \emph{RBox} $\Rmc$ is a finite set of \emph{role inclusions} (RIAs) of the form $R_1 \circ \cdots \circ R_n \sqsubseteq R$, and disjoint role axioms $\disjoint(S_1, S_2)$ where $R$, $R_1$, $\dots$, $R_n$, $S_1$, and $S_2$ are roles. $S_1$ and $S_2$ are simple (defined next) in the RBox $\Rmc$. The special case of $n = 1$ is a \emph{simple role inclusion}, while we call the cases where $n > 1$ \emph{complex role inclusions}. The RBox represents knowledge about the relationships between roles.

% describe simple and complex roles since it is relevant
The set of \emph{non-simple} roles in $\Rmc$ is the smallest set such that: $U$ is non-simple; any role $R$ that appears as the super role of a complex RIA $R_1 \circ \cdots \circ R_n \sqsubseteq R$ where $n > 1$ is non-simple; any role $R$ that appears on the right-hand side of a simple RIA $S \sqsubseteq R$ where $S$ is non-simple, is also non-simple; and a role $r$ is non-simple if and only if $r^-$ is non-simple.
All other roles are \emph{simple}.

% describe regularity since it is relevant
For convenience, let us define the function $\inv(R)$ such that $\inv(U) = U$, $\inv(r) = r^-$, and $\inv(r^-) = r$ for all role names $r \in N_R$. 
%
An RBox $\Rmc$ is \emph{regular} if there exists a preorder $\preceq$, i.e., a transitive and reflexive relation, over the set of roles appearing in $\Rmc$, such that $R \preceq S \iff \inv(R) \preceq S$, are of the forms:
$\inv(R) \sqsubseteq R$,
$R \circ R \sqsubseteq R$,
$S \sqsubseteq R$, $R \circ S_1 \circ \cdots \circ S_n \sqsubseteq R$,
$S_1 \circ \cdots \circ S_n \circ R \sqsubseteq R$, or
$S_1 \circ \cdots \circ S_n \sqsubseteq R$,
where $n > 1$ and $R$, $S$, $S_1, \cdots, S_n$ are roles such that $S \preceq R$, $S_i \preceq R$, and $R \not\preceq S_i$ for $i = 1, \dots, n$. This regularity restriction has been chosen to align with the implementation of the OWL 2 DL \cite{motik2012ontology} profile checker in the OWL API \cite{horridge2011owl}.

% TODO argue for my version of regularity

\begin{definition}
A \SROIQ  ontology $\Omc = \Tmc \cup \Amc \cup \Rmc$ consists of a TBox $\Tmc$, an ABox $\Amc$, and a RBox $\Rmc$ in the language of \SROIQ, and where the RBox $\Rmc$ is regular.
\end{definition}

\subsection{The Semantics of \SROIQ} \label{sroiq-semantics}

% describe the semantics
% whats a interpretation, when is it a model
The semantics of \SROIQ is defined using \emph{interpretations} $I = \langle \Delta^I, \cdot^I \rangle$, where $\Delta^I$ is a non-empty \emph{domain} and $\cdot^I$ is a function associating to each individual name $a$ an element of the domain $a^I \in \Delta^I$, to each concept $C$ a subset of the domain $C^I \subseteq \Delta^I$, and to each role $R$ a binary relation on the domain $R^I \subseteq \Delta^I \times \Delta^I$. The interpretation function $\cdot^I$ is extended to the interpretation of complex roles and concepts, such that
\begin{align*}
  U^I &= \Delta^I \times \Delta^I \enspace, &
  \top^I &= \Delta^I \enspace, &
  \bot^I &= \emptyset \enspace, &
  \nominal{a}^I &= \{ a^I \} \enspace,
\end{align*}
\vspace{-9mm}
\begin{align*}
  (\lnot C)^I &= \Delta^I \setminus C^I \enspace, &
  (C \sqcap D)^I &= C^I \cap D^I \enspace, &
  (C \sqcup D)^I &= C^I \cup D^I \enspace,
\end{align*} 
\vspace{-9mm}
\begin{align*}
  (r^-)^I &= \{ \langle \gamma_2, \gamma_1 \rangle \mid \langle \gamma_1, \gamma_2 \rangle \in r^I \} \enspace, \\
  (\forall R.C)^I &= \{ \gamma \in \Delta^I \mid \forall \gamma' \in \Delta^I : \langle \gamma, \gamma' \rangle \in R^I \implies \gamma' \in C^I \} \enspace, \\
  (\exists R.C)^I &= \{ \gamma \in \Delta^I \mid \exists \gamma' \in \Delta^I : \langle \gamma, \gamma' \rangle \in R^I \text{ and } \gamma' \in C^I \} \enspace, \\
  (\less n {R.C})^I &= \{ \gamma \in \Delta^I \mid \left| \{ \gamma' \in \Delta^I \mid \langle \gamma, \gamma' \rangle \in R^I \text{ and } \gamma' \in C^I \} \right| \leq n \} \enspace, \\
  (\more n {R.C})^I &= \{ \gamma \in \Delta^I \mid \left| \{ \gamma' \in \Delta^I \mid \langle \gamma, \gamma' \rangle \in R^I \text{ and } \gamma' \in C^I \} \right| \geq n \} \enspace, \\
  (\exists R.\self)^I &= \{ \gamma \in \Delta^I \mid \langle \gamma, \gamma \rangle \in R^I \} \enspace,
\end{align*}
where $|X|$ is the cardinality of a set $X$.
An interpretation $I$ is a \emph{model for} $\Omc$, written $I \models \Omc$, if it satisfies all the axioms in $\Omc$. Whether the interpretation $I$ satisfies the axiom $\alpha$, written $I \models \alpha$, is given by
\begin{align*}
  I \models C \sqsubseteq D &\iff C^I \subseteq D^I \enspace, &
  I \models R(a, b) &\iff \langle a^I, b^I \rangle \in R^I \enspace,\\
  I \models C(a) &\iff a^I \in C^I \enspace, &
  I \models \lnot R(a, b) &\iff \langle a^I, b^I \rangle \not\in R^I \enspace, \\
  I \models a = b &\iff a^I = b^I \enspace, &
  I \models \disjoint(R, S) &\iff R^I \cap S^I = \emptyset \enspace, \\
  I \models a \not= b &\iff a^I \not= b^I \enspace, &
  I \models S_1 \circ \cdots \circ S_n \sqsubseteq R &\iff S_1^I \circ \cdots \circ S_n^I \subseteq R^I \enspace,
\end{align*}
where $\circ$ denotes the composition of relations, that is, $r \circ s = \{ \langle x, y \rangle \mid \exists z : \langle x, z \rangle \in r \text{ and } \langle z, y \rangle \in s \}$.

An ontology $\Omc$ is \emph{consistent} if there exists a model $I \models \Omc$ for the ontology. An ontology that is not consistent is \emph{inconsistent}. A concept $C$ is \emph{satisfiable} in an ontology $\Omc$ if there exists a model $I \models \Omc$ in which $C^I \not= \emptyset$. An ontology in which all atomic concepts $C \in N_C$ are satisfiable is called \emph{coherent}. We say an axiom $\alpha$ is \emph{entailed by} the ontology $\Omc$, written $\Omc \models \alpha$, if and only if every model $I \models \Omc$ of the ontology satisfies $I \models \alpha$. 

% describe subsumptions of concepts and roles
Given two concepts $C$ and $D$ we say that $C$ is \emph{subsumed} by $D$ (or $D$ \emph{subsumes} $C$) with respect to the ontology $\Omc$, written $C \sqsubseteq_\Omc D$, if $C^I \subseteq D^I$ in every model $I$ of $\Omc$. Further, $C$ is \emph{strictly subsumed by} $D$, written $C \sqsubset_\Omc D$, if $C \sqsubseteq_\Omc D$ but not $D \sqsubseteq_\Omc C$. Analogously, given two roles $R$ and $S$, $R$ is subsumed by $S$ with respect to $\Omc$ ($R \sqsubseteq_\Omc S$) if $R^I \sqsubseteq S^I$ in all models $I$ of $\Omc$. Again, $R \sqsubset_\Omc S$ holds if $R \sqsubseteq_\Omc S$ but not $D \sqsubseteq_\Omc C$.

\subsection{The \ALC Description Logic}

Since \ALC will be used for part of the discussion on related work, this is a brief section discussing what features are included in \ALC, and which are not. \ALC is a much less expressive logic compared to \SROIQ. The vocabulary in \ALC consists of the same components $N_C$, $N_R$, and $N_I$ as in \SROIQ. In \ALC, concept expressions are formed only using the following grammar.
\begin{align*}
  C &::= \bot \mid \top \mid A \mid \neg C \mid C \sqcap C \mid C \sqcup C \mid \forall R.C \mid \exists R.C \enspace,
\end{align*}
where $A \in N_C$ is a concept name and $r \in N_R$ is a role name. The universal role or inverse roles are not part of \ALC. Further, \ALC supports only two types of axioms, GCIs of the form $C \sqsubseteq D$ and class assertions $C(a)$ where $C$ and $D$ are concepts and $a \in N_I$ is an individual name. It should be noted that \ALC has the unique name assumption, meaning that two different individual names must necessarily refer to different elements of the domain in every interpretation. Aside from that, the axioms and concepts have the same semantic meaning as their direct counterparts in \SROIQ.



As software systems evolve, it becomes harder to avoid the introduction of bugs. Similarly, in ontology engineering, bugs can be introduced into an ontology. With increased size and complexity of a system, it becomes harder to debug these defects, both for software systems and ontologies.

\subsection{Categories of Bugs} \label{categories-of-bugs}

Defects, in both software systems and ontologies, can be due to a number of different reasons. In \cite{kalyanpur2005debugging} the authors identify three broad categories of defects that can be present in an ontology: \emph{syntactic defects}, \emph{semantic defects,} and \emph{modelling defects}.

\subsubsection{Syntactic Defects} \label{syntactic-defects}

Syntactic defects in an ontology can be caused by a statement that does not conform to the grammar of the employed logic. Similarly, for software systems, these defects may be the result of programs that are not consistent with the grammar of the chosen programming language. These sorts of syntactic defects are easy to locate and correct. In general, tool support for these kinds of defects is able to pinpoint the location of the defect and give an explanation to the user.

\begin{example}
  The axioms $C \sqsubseteq \sqsubseteq D$, $C(r(a, b))$, or $C \sqsubset D$ are all obviously not syntactically correct \SROIQ axioms. Similarly, the concepts $C \sqcup \sqcap D$, $\less 3 t.\self$, and $\self \sqcup D$ do all not match the grammar for concept expressions.
\end{example}

There may however be some additional restrictions on what constitutes a valid ontology or program that is not based solely on the grammatical rules. For ontologies, these might be for example the restrictions placed upon the form of the graph for a specific OWL profile. In the example of \SROIQ, these include the restricted use of non-simple roles and the regularity condition placed on the RBox. For programming languages, a similar restriction to this may be the requirement for definition before use or the presence of a type system. These restrictions reduce the space of valid programs. Restrictions of this kind can often be much easier to violate and harder to debug than the first kind of syntactic defects that is a violation of the grammar.

\begin{example}
  Given the vocabulary $N_C = \{ C \}$ and $N_R = \{ a, b, c \}$, the \SROIQ ontology $\Omc = \{ a \sqsubseteq b, b \circ c \sqsubseteq a \}$ is not valid. This is because it does not satisfy the regularity condition in \SROIQ. This can simply be verified by showing that no preorder $\preceq$ exists for which $a \preceq b$ and $a \not\preceq b$ hold.

  Similarly, the ontology $\Omc = \{ a \circ a \sqsubseteq a, \top \sqsubseteq \exists a.\self \}$ is not valid because the role $a$ is non-simple. Since roles used in \self constraints must be simple, the second axiom is not allowed.
\end{example}

\subsubsection{Semantic Defects} \label{semantic-defects}

For ontologies, semantic defects, as defined in \cite{kalyanpur2005debugging}  are those which can be discovered by a reasoner given an ontology free of syntactic defects. This includes for example the inconsistency of the ontology, or the unsatisfiability of a concept. The presence of such defects is generally not hard to identify, given the availability of a reasoner for the logic of the ontology. It is, however, often not trivial to understand the underlying source of the defect.

\begin{example}
  Given the vocabulary $N_C = \{ C, D \}$ and $N_I = \{ a \}$, the \SROIQ ontology $\Omc = \{ D \sqsubseteq \lnot C, D(a), C(a) \}$ is inconsistent. This can be seen clearly from the fact that for every interpretation $I$, $a^I$ must be in $D^I$, but this means that $a^I$ must not be in $C^I$ which contradicts with the last axiom.
\end{example}

A close analogy to these kinds of defects from the perspective of a software system is the raising of an error during the execution. An error is an indication of a defect in the software, and depending on tooling support they may be more or less difficult to understand and rectify. In this thesis the focus will be mainly on these kinds of bugs, especially on the problem of inconsistent ontologies.

\subsubsection{Modelling Defects} \label{modelling-defects}

Modelling defects are those defects that are not syntactically or semantically invalid. The presence of unintended inferences in an ontology is one such defect. These defects can also be of more stylistic nature. Redundancy or unused parts of the ontology may be considered as defects, since they do not add any knowledge to the ontology.

For software systems, modelling defects are bugs that do not cause any errors, but which produce undesired behaviour. An example for such a defect could be that the result of a calculation is wrong, or that the software includes security vulnerabilities. For software systems, there might be other non-functional requirements, that if not met constitute defects in the software. These may for example be unsatisfactory performance or unmaintainable code organization.

These kinds of defects can in general not be detected automatically by tools. They require careful attention and domain specific knowledge to be revealed and corrected. In some scenarios, testing may be used to uncover and prevent against some modelling defects, by expressing more explicitly the modellers or programmers intention. This can be done both for software systems and for ontologies.

\subsection{Causes of Bugs} \label{causes-of-bugs}

We will now briefly explore the possible causes of bugs in the context of ontology engineering. One major source of defects is of course human error on part of the ontology developer. These can range from trivial typos, that will most often simply result in syntactic errors, or from fundamental misunderstandings of the domain, which can lead to modelling mistakes. Especially syntactic errors concerning the additional constraints, like regularity and simplicity, can easily be violated as the result of an oversight by the modeller. One can for example define a role as transitive, and then afterwards use it in cardinality constraints. Unfortunately, however, this is not allowed in many ontology formalisms, such as in \SROIQ or OWL 2 DL. Similarly, mistakes during the modelling process can cause inconsistent ontologies or unsatisfiable concepts. The modeller might have defined two classes as disjoint, but later added an individual belonging to the intersection of the two concepts, yielding an inconsistent ontology.

Another source for bugs in ontologies may arise while merging different ontologies. This may be done, especially in the context of OWL and the semantic web, to combine the knowledge obtained from different sources. OWL even includes an axiom that can be used to import other ontologies. This can again be a case where one ontology defines a certain role as transitive, while the other assumes that it is simple. Combining the two ontologies will then obviously lead to a violation of the global constraints. In general, the combination of different ontologies can also change which profile the ontologies fall into. Also, different ontologies might make different but incompatible modelling choices when modelling the same domain. This can result is an ontology after merging that is inconsistent or has unsatisfiable concepts.

\begin{example}
  Given the vocabulary $N_C = \{ C, D \}$ and $N_I = \{ a, b \}$, the two \SROIQ ontology $\Omc_1 = \{ D \sqsubseteq C, D(a) \}$ and $\Omc_2 = \{ C \sqsubseteq \lnot D, C(b) \}$ are by themselves consistent. If we merge them, however, the resulting ontology $\Omc_1 \cup \Omc_2$ is no longer consistent.
\end{example}

% TODO make this shorter


As software systems evolve, it becomes harder to avoid the introduction of bugs. Similarly, in ontology engineering, bugs can be introduced into an ontology. With increased size and complexity of a system, it becomes harder to debug these defects, both for software systems and ontologies.

\subsection{Categories of Bugs} \label{categories-of-bugs}

Defects, in both software systems and ontologies, can be due to a number of different reasons. In \cite{kalyanpur2005debugging} the authors identify three broad categories of defects that can be present in an ontology: \emph{syntactic defects}, \emph{semantic defects,} and \emph{modelling defects}.

\subsubsection{Syntactic Defects} \label{syntactic-defects}

Syntactic defects in an ontology can be caused by a statement that does not conform to the grammar of the employed logic. Similarly, for software systems, these defects may be the result of programs that are not consistent with the grammar of the chosen programming language. These sorts of syntactic defects are easy to locate and correct. In general, tool support for these kinds of defects is able to pinpoint the location of the defect and give an explanation to the user.

\begin{example}
\end{example}


\begin{example}
\end{example}

\subsubsection{Semantic Defects} \label{semantic-defects}

For ontologies, semantic defects, as defined in \cite{kalyanpur2005debugging}  are those which can be discovered by a reasoner given an ontology free of syntactic defects. This includes for example the inconsistency of the ontology, or the unsatisfiability of a concept. The presence of such defects is generally not hard to identify, given the availability of a reasoner for the logic of the ontology. It is, however, often not trivial to understand the underlying source of the defect.

\begin{example}
\end{example}


\subsubsection{Modelling Defects} \label{modelling-defects}

Modelling defects are those defects that are not syntactically or semantically invalid. The presence of unintended inferences in an ontology is one such defect. These defects can also be of more stylistic nature. Redundancy or unused parts of the ontology may be considered as defects, since they do not add any knowledge to the ontology.

For software systems, modelling defects are bugs that do not cause any errors, but which produce undesired behaviour. An example for such a defect could be that the result of a calculation is wrong, or that the software includes security vulnerabilities. For software systems, there might be other non-functional requirements, that if not met constitute defects in the software. These may for example be unsatisfactory performance or unmaintainable code organization.

These kinds of defects can in general not be detected automatically by tools. They require careful attention and domain specific knowledge to be revealed and corrected. In some scenarios, testing may be used to uncover and prevent against some modelling defects, by expressing more explicitly the modellers/programmers intend. This can be done both for software systems and for ontologies.

\subsection{Causes of Bugs} \label{causes-of-bugs}

% TODO

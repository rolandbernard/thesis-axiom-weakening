
The main difficulties that arise when weakening axioms in \SROIQ ontologies, and especially when weakening RIAs, are related to ensuring that the constraints on the use of non-simple roles and the regularity of the RBox as a whole are maintained. In many logics, the union of two ontologies would be guaranteed to be another valid ontology. While this is the case for FOL theories or even \ALC ontologies, it is not true for \SROIQ ontologies. As mentioned in \cref{causes-of-bugs}, merging two \SROIQ ontologies may create an ontology that violates the constraints. While these restrictions help ensure the decidability of the logic, they also make them more difficult to use. Not only must these problems be taken into account when authoring ontologies, but they must be carefully considered also when modifying them using automatic transformations, such as applying axiom weakening based repair algorithms. Replacing an axiom with another axiom, even if the new axiom is weaker, can cause the resulting ontology to violate one or more of the global constraints in \SROIQ. This may happen even if the new axiom on its own is valid, as can be seen in \cref{ex:bad-weakening}.

\begin{example} \label{ex:bad-weakening}
  Take the ontology $\Omc = \{ r \circ s \circ r \sqsubseteq t, r \sqsubseteq s, \top \sqsubseteq \forall t.\bot, \exists s.\self \sqsubseteq \top \}$. The extension of $t$ is empty in every model of this ontology. It follows that, ignoring the global constraints, the axiom $r \sqsubseteq s$ could be weakened to $t \sqsubseteq s$. This would, however, result in an ontology where $s$ is non-simple, which is not allowed since $s$ is used as part of a \self constraint.
  Further, the RBox of the resulting ontology will not be regular, since for any preorder $\preceq$, $t \preceq s$ must hold for the new axiom, but $t \not\preceq s$ is required by the complex RIA.
\end{example}

\begin{example}
  As another example, with less trivial weakening, take the ontology $\Omc = \{ r \circ s \circ r \sqsubseteq t, \top \sqsubseteq \forall t.\nominal a, \top \sqsubseteq \exists s.\nominal a \}$. Since the range of $t$ is restricted to the single individual $a$, and $s$ contains all connections to $a$, $t \sqsubseteq_\Omc s$. The axiom $r \circ s \circ r \sqsubseteq t$ could therefore be weakened to $r \circ s \circ r \sqsubseteq s$. Yet, this would result in a non-regular RBox.
\end{example}

The weakening proposed in \cite{troquard2018repairing} and described above in \cref{weakening-alc} operates on \ALC ontologies. A similar approach can be used also for more expressive logics like \ALCHOIQ, without additional constraints. In order to extend axiom weakening to \SROIQ ontologies, however, it is necessary to restrict the allowed weakening accordingly, to ensure that these issues are avoided. In \cite{confalonieri2020towards} an axiom weakening operator is proposed for parts of \SROIQ. However, to prevent the aforementioned issues, some restrictions are put in place. For example, the refinement of RIAs has not been considered at all. For this thesis, however, the axiom weakening operator has been extended to handle also RIAs. To achieve this, it must be ensured that when weakening disjoint role axioms or refining cardinality and \self constraints only simple roles are used. Further, it must be guaranteed that all roles that are currently used in such contexts remain simple even after adding the weakened axioms to the ontology. Finally, the regularity of the RBox must be maintained by the addition of the weakened axioms. We now discuss which restrictions we applied in order to satisfy these requirements.

Firstly, the covers and refinement operators for roles are defined to operate only on simple roles. A similar restriction has already been applied to the refinement operators suggested in \cite{confalonieri2020towards}. Restricting the refinement to simple roles guarantees that the new axioms created by weakening will not contain non-simple roles in axioms or concepts where simple roles are required. An important detail that has not been considered in \cite{confalonieri2020towards} is that the roles over which the covers operate must be simple in all ontologies that the weaker axioms will be used in. It is therefore generally not sufficient to use all the roles that are simple in the reference ontology, since the reference ontology may not contain all RBox axioms, and therefore contain simple roles that are not simple in the full ontology.

\begin{example}
  Given the inconsistent ontology $\Omc = \{ s \circ s \sqsubseteq s, s \sqsubseteq r, t \sqsubseteq r, \top \sqsubseteq \forall r.\bot, \top \sqsubseteq \exists s.\top, \top \sqsubseteq \exists t.\self \}$, we may choose as reference ontology the maximal consistent subset $\Omcref = \Omc \setminus \{ s \sqsubseteq r \}$. In this case, $r$ is simple in $\Omcref$, however, it is not allowed to weaken $\top \sqsubseteq \exists t.\self$ to $\top \sqsubseteq \exists r.\self$, because $r$ is not simple in $\Omc \setminus \{ \top \sqsubseteq \exists t.\self \} \cup \{ \top \sqsubseteq \exists r.\self \}$ and can therefore not be used in \self constraints.
\end{example}

For this reason, we give to the upward and downward cover as an argument not only the reference ontology $\Omcref$, but also the full ontology $\Omcfull$. The same must be done for the axiom weakening operator that uses the cover functions. Both $\Omcref$ and $\Omcfull$ share the same vocabulary $N_I$, $N_C$, and $N_R$. We assume that all roles that are simple in $\Omcfull$ are also simple in $\Omcref$. While this is not strictly required, it will be true for all algorithms shown in this thesis that $\Omcref \subseteq \Omcfull$. In the context of repairing ontologies, $\Omcfull$ can be chosen to be the unmodified ontology that we want to repair.

An alternative to always choosing simple roles is to define two versions of the covers, one which operators only over simple roles and one which uses also non-simple roles. Then in a context in which simple roles are required, the more restrictive cover is used, while in all other cases the complete covers are used. While this variation has been implemented, it has not been evaluated and will not be further considered in the rest of the thesis.

Another assumption made in \cite{confalonieri2020towards} is that the existential and universal roles are simple, and may therefore be included in the upward and downward covers. While this is not a significant restriction in theory, it contradicts the OWL 2 DL specification, which directly defines these roles to be non-simple in every ontology. Including the existential and universal role in the covers, while augmenting the number of possible weakening, is not necessarily required. Therefore, this thesis will consider, as stated in \cref{sroiq-syntax}, the universal role to be non-simple. This allows the implementation of the proposed approach using existing highly-optimized reasoners designed for OWL 2 DL. It has to be considered, however, that removing the existential and universal role from the covers can prevent weakening of RIAs, disjoint role axioms and role assertions to tautologies. Removal of axioms may be required if they are by themselves inconsistent, meaning that not being able to weaken them to tautologies could mean the repair algorithm never terminates. For this reason, the weakening operator for these axioms explicitly includes a tautological axiom. Note that for the case of positive and negative role assertions, the extension of the covers to include also non-simple roles would be an alternative solution. This is however not the case for disjoint role assertions, as the roles used in them must always be simple.


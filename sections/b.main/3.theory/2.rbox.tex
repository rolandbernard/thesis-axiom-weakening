
Let us now consider the weakening of simple and complex RIAs. To ensure that by adding weakened axioms we do not cause a constraint violation in existing axioms and concepts, we choose the allowed weakening for RIAs such that all roles that are simple in $\Omcfull$, are also simple after adding to it a weakening of one of its axioms. We observe that for complex RIAs $S_1 \circ \cdots \circ S_n \sqsubseteq R$ we should not refine the role $R$. Since all roles returned by our refinement operator will be simple in $\Omcfull$, replacing $R$ with a refinement $R'$ would create a new axiom which, when added to the ontology, would make the role $R'$ which was simple in $\Omcfull$ non-simple. A similar argument can be made for refining $R$ in a simple RIA $S \sqsubseteq R$ where the role $S$ is non-simple in $\Omcfull$. Therefore, The only case in which it is possible to refine the super role of a RIA during axiom weakening requires that the RIA is simple and additionally that the sub role of the RIA is simple in $\Omcfull$.

When it comes to refining the left-hand side of RIAs, we do not need any special restrictions. The important observation here is that all roles that are returned by the refinement will be simple. This means that in a simple RIA $R \sqsubseteq S$, even if $S$ is simple, replacing $R$ with another simple role will not cause $S$ to become non-simple. For a complex RIA $S_1 \circ \cdots \circ S_n \sqsubseteq R$ on the other hand, the role $R$ must already have been non-simple in $\Omcfull$, and replacing any $S_i$ with a refinement has no effect on whether a role is simple or non-simple.

A more interesting question is whether such a weakening may still cause a non-regular RBox. The main insight is that simple roles are always allowed on the left-hand side of a RIA. While this is more directly evident in some alternative definitions of regularity (e.g., \cite{rudolph2011foundations}) it is not so apparent from the one presented in this thesis. Intuitively, the constraints given in \cref{sroiq-syntax} for regularity disallow dependency cycles that contain complex RIAs. Simple roles cannot be part of such a cycle, since the cycle must contain at least one complex RIA to be a violation of the constraint, and all roles that depend in this sense on a complex RIA must be non-simple. A more formal justification for this fact is given in the \hyperref[proof:regularity]{proof} for \cref{lem:regularity}. Since all refinements of the left-hand side of RIAs are performed using simple roles, these cannot lead to a non-regular RBox. Further, refinements of the super role of RIAs are only performed on simple RIAs $S \sqsubseteq R$ where $S$ is a simple role. Since $S$ is simple in this case, all refinements of $R$ are allowed, potentially also if the refinement yielded a non-simple role.

Note that when using this approach to weakening RIAs in an ontology in which all roles are simple, such as in \ALCH, we get the obvious weakening operator that specializes the left-hand side and generalizes the right-hand side, without any further restrictions.

\subsection{An Alternative Approach}\label{rbox-alternative}

Instead of limiting weakening of RIAs in this way, an alternative approach that may be considered is to generate axioms and then filter out those that would violate some restrictions. The axioms can be generated by using the specialization operator on all roles of the left-hand side and the specialization operator to the role on the right-hand side. Say we still want to keep all simple roles of $\Omcfull$ simple after addition of the weakened RIA. Then we would filter out all complex RIA in which the right-hand side became a role that is supposed to be simple. Further, all simple RIA in which the left-hand side is not guaranteed to be simple, but the right-hand side must be, should be removed.

Additionally, to guarantee the regularity of the RBox, a preorder $\preceq$ may be fixed before the start of the repair. The preorder must be a witness for the regularity of the ontology $\Omcfull$. Then, when weakening RIAs, they must be filtered to conform to the preorder $\preceq$. This approach will ensure regularity, since all RIAs that may be added will conform to the chosen preorder. All existing RIAs trivially conform to it, given it is a precondition for the choice of $\preceq$. One important detail is that the preorder may not be changed in between a repair. It is not generally safe to combine a weakened RIA generated using one preorder with another RIA that used another preorder during weakening. This mechanism has been implemented, but not evaluated or further explored in this thesis.

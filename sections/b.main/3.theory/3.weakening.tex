
We will now give the extended definitions for the cover sets, refinement operators, and axiom weakening operator in \SROIQ. 

\begin{lemma}\label{lem:basic}
  For every pair of \SROIQ ontologies $\Omcref, \Omcfull$ and every pair of concepts or roles $X, Y \in \Lmc(N_C, N_R, N_I) \cup \Lmc(N_R)$:
  \newcommand\litem[1]{\item{\bfseries #1:\enspace }}
  \begin{enumerate}
    \litem{generalisation}\label{lem:generalisation} if $X \in \gamma_{\Omcref,\Omcfull}(Y)$ then $Y \sqsubseteq_{\Omcref} X$ \\
    \textbf{specialisation:\enspace} if $X \in \rho_{\Omcref,\Omcfull}(Y)$ then $X \sqsubseteq_{\Omcref} Y$
    \litem{generalisation finiteness} $\gamma_{\Omcref,\Omcfull}(X)$ is finite \\
    \textbf{specialisation finiteness:\enspace} $\rho_{\Omcref,\Omcfull}(X)$ is finite
  \end{enumerate}
\end{lemma}

\begin{proof}
% TODO
\end{proof}


\begin{lemma} \label{lem:weaker}
  For every \SROIQ axiom $\phi$, if $\phi' \in g_{\Omcref,\Omcfull}(\phi)$, then $\phi \models_\Omcref \phi'$
\end{lemma}

\begin{proof} We will handle each type of axiom separately.
  \begin{itemize}
    \item If $\phi = C \sqsubseteq D$, suppose $\phi' = C' \sqsubseteq D'$. From \cref{lem:basic}.\ref{lem:generalisation} we know that $C' \sqsubseteq_\Omcref C$ and $D \sqsubseteq_\Omcref D'$. By transitivity of subsumption, we conclude that $C \sqsubseteq D \models_\Omcref C' \sqsubseteq D'$.
    \item If $\phi = C(a)$, suppose $\phi' = C'(a)$. From \cref{lem:basic}.\ref{lem:generalisation} we know that $C \sqsubseteq_\Omcref C'$. Given any model $I$ of $\Omcref \cup \{ \phi \}$, $a^I \in C^I$. Since $C^I \subseteq C'^I$ in every model of $\Omcref$, $a^I \in C'^I$. We conclude that $C(a) \models_\Omcref C'(a)$.
    \item If $\phi = R(a, b)$, suppose $\phi' = R'(a, b)$. From \cref{lem:basic}.\ref{lem:generalisation} we know that $R \sqsubseteq_\Omcref R'$. Given any model $I$ of $\Omcref \cup \{ \phi \}$, $\langle a^I, b^I \rangle \in R^I$. Since $R^I \subseteq R'^I$ in every model of $\Omcref$, $\langle a^I, b^I \rangle \in R'^I$. We conclude that $R(a, b) \models_\Omcref R'(a, b)$.
    \item If $\phi = \lnot R(a, b)$, suppose $\phi' = R'(a, b)$. From \cref{lem:basic}.\ref{lem:generalisation} we know that $R' \sqsubseteq_\Omcref R$. Given any model $I$ of $\Omcref \cup \{ \phi \}$, $\langle a^I, b^I \rangle \not\in R^I$. Since $R'^I \subseteq R^I$ in every model of $\Omcref$, $\langle a^I, b^I \rangle \not\in R'^I$. We conclude that $\lnot R(a, b) \models_\Omcref \lnot R'(a, b)$.
    \item If $\phi = \disjoint(R, S)$, suppose $\phi' = \disjoint(R', S')$. From \cref{lem:basic}.\ref{lem:generalisation} we know that $R' \sqsubseteq_\Omcref R$ and $S' \sqsubseteq_\Omcref S$. Given any model $I$ of $\Omcref \cup \{ \phi \}$, $R^I \cap S^I = \emptyset$. Since $R'^I \subseteq R^I$ and $S'^I \subseteq S^I$ in every model of $\Omcref$, $R'^I \cap S'^I = \emptyset$. We conclude that $\disjoint(R, S) \models_\Omcref \disjoint(R', S')$.
    \item If $\phi = S_1 \circ \cdots \circ S_n \sqsubseteq R$, suppose $\phi' = S_1' \circ \cdots \circ S_n' \sqsubseteq R'$. From \cref{lem:basic}.\ref{lem:generalisation} we know that $R \sqsubseteq_\Omcref R'$ and $S_i' \sqsubseteq_\Omcref S_i$ for $i = 1, \dots, n$. Given any model $I$ of $\Omcref \cup \{ \phi \}$, $S_1^I \circ \cdots \circ S_n^I \subseteq R^I$. Since $R^I \subseteq R'^I$ and $S_i'^I \subseteq S_i^I$ for $i = 1, \dots, n$ in every model of $\Omcref$, $S_1'^I \circ \cdots \circ S_n'^I \subseteq R'^I$. We conclude that $S_1 \circ \cdots \circ S_n \sqsubseteq R \models_\Omcref S_1' \circ \cdots \circ S_n' \sqsubseteq R'$.
  \end{itemize}
\end{proof}


\begin{lemma} \label{lem:simple-roles}
  Let $\Omc$ be an ontology such that all simple roles of $\Omcfull$ are also simple in $\Omc$. For every axiom $\phi \in \Omc$ and role $R$, if $\phi' \in g_{\Omcref,\Omcfull}(\phi)$ and $R$ simple in $\Omc$, then $R$ is simple in $\Omc \cup \{ \phi' \}$.
\end{lemma}

\begin{proof}(\emph{Sketch})
For the addition to change the simplicity of any role, it must be that $\phi'$ is a RIA that has some role $R$ as the super role and is either complex, or for which the sub role is non-simple. Assume, by contradiction, that $R$ is a simple role in $\Omc$ and non-simple in $\Omc \cup \{ \phi' \}$. Since $R$ is simple in $\Omc$ it is not the universal, does not appear as the super role in any complex RIA of $\Omc$, and neither on the right-hand side of a simple RIA in $\Omc$ where the sub role is non-simple. If $\phi'$ is a complex RIA then, by definition of the weakening operator, $\phi$ must be a complex RIA and $R$ must also be the super role in $\phi$, making it non-simple in $\Omc$, which contradicts our assumption. Similarly, if $\phi'$ is a simple RIA with a non-simple role as the sub role, the sub role of $\phi$ must be equal to that of $\phi'$ because the refinement operators return only roles simple in $\Omcfull$, and those are also simple in $\Omc$. Further, since the super role of a RIA is only refined if the sub role is simple, $\phi' = \phi$, which means that $R$ is non-simple in $\Omc$, which contradicts the assumptions. It follows that such a role $R$ does not exist.
\end{proof}



\begin{lemma} \label{lem:regularity}
  Let $\Omc$ be an ontology such that all simple roles of $\Omcfull$ are also simple in $\Omc$. For every axiom $\phi \in \Omc$, if $\phi' \in g_{\Omcref,\Omcfull}(\phi)$ and the RBox of $\Omc$ is regular, then the RBox of $\Omc \cup \{ \phi' \}$ is also regular.
\end{lemma}

\begin{proof}(\emph{Sketch}) \phantomsection\label{proof:regularity}
Let us first argue that if there exists a preorder that satisfies the constraints necessary for checking regularity, then there exists $\preceq$ such that $S_1 \preceq S_2$, $S \preceq R$ and $R \not\preceq S$ for all simple roles $S, S_1, S_2$ and non-simple roles $R$. Firstly, $S_1 \not\preceq S_2$ and $S \not\preceq R$ cannot be required, because absence of a tuple is only required for complex RIAs, where the super role must not be a predecessor of the roles on the left-hand side. Since $S_1$ and $S$ are simple, they do not appear as the super role in a complex RIA. Similarly, $R \preceq S$ cannot be required. Since $S$ is simple and $R$ non-simple, it cannot be required directly through an axiom of the form $R \sqsubseteq S$. By induction, it cannot be required through transitivity, since $R \preceq T$ and $T \preceq S$ would have to be required. If $T$ is simple, $R \preceq T$ cannot be required, and if $T$ is non-simple, $T \preceq S$ cannot be required.

Since $\Omc$ has a regular RBox, there exists such a $\preceq$ for $\Omc$. We will show that $\preceq$ is also a witness for regularity of $\Omc \cup \{ \phi' \}$. All RIA in $\Omc$ are of one of the allowed forms for $\preceq$. It is therefore sufficient to verify that $\phi'$ has one of the allowed forms.
If $\phi' = \phi$ or $\phi'$ is not a RIA, it does not affect the regularity.
Otherwise, if $\phi'$ is a simple RIA $S \sqsubseteq R$, then by definition of the weakening operator, $S$ is simple in $\Omcfull$, and therefore also in $\Omc$. Given that $S$ is simple, $S \preceq R$ holds for simple and non-simple $R$ by our choice of $\preceq$.
If $\phi'$ is a complex RIA $S_1' \circ \cdots \circ S_n' \sqsubseteq R$, then $\phi$ is also a complex RIA $S_1 \circ \cdots \circ S_n \sqsubseteq R$ and $R$ is non-simple in $\Omc$. If $S_i \preceq R$ and $S_i \not\preceq R$, then so will $S_i' \preceq R$ and $R \not\preceq S_i'$, either because $S_i = S_i'$ or because $S_i'$ is simple and $R$ is non-simple. Since $\Omc$ has a regular RBox, the only case in which $R \preceq S_i$ is if $S_i = R$. In this case, $S_i' \preceq R$ and $R \not\preceq S_i'$ will still hold if $S_i \not= R$. If $S_i = R$, either $i = 0$ or $i = n$ which is allowed. The only delicate case is if $\phi = R \circ R \sqsubseteq R$, which will result in either $\phi' = S_1' \circ R \sqsubseteq R$ or $R \circ S_2' \sqsubseteq R$, both of which are valid.
\end{proof}



\begin{theorem} \label{lem:global-constraints}
  Given that $\Omcref$ and $\Omcfull$ are valid \SROIQ ontologies. For every axiom $\phi \in \Omcfull$, if $\phi' \in g_{\Omcref,\Omcfull}(\phi)$, then $\Omcfull \cup \{ \phi' \}$ is a valid \SROIQ ontology.
\end{theorem}

\begin{proof}(\emph{Sketch})
We have established already in \cref{lem:regularity}, that the regularity of the RBox will be preserved.
It is guaranteed by \cref{lem:simple-roles} that all roles that were simple before addition, are still simple afterwards. Therefore, all usages of roles in axioms and concepts that were not touched by the refinement do not pose a problem. The condition static that the upcover and downcover of a role contain only roles that are simple in $\Omcfull$ (and therefore by \cref{lem:simple-roles} also in $\Omcfull \cup \{ \phi' \}$) forces that every refinement of a role is simple. This restriction to simple roles guarantees that no non-simple role may be used in disjoint role axioms, or the scope of cardinality and self constraints.
\end{proof}

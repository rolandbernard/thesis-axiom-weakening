\documentclass[aspectratio=169]{beamer}
\usefonttheme{serif}
\renewcommand\familydefault{\sfdefault} 

\usepackage{graphicx}
\usepackage[english]{babel}
\usepackage[T1]{fontenc}
\usepackage{pdfx, csquotes, xcolor, xspace, changepage, setspace}
\usepackage{libertinus}
\usepackage{roboto}
\usepackage{fontawesome5}
\usepackage[maxbibnames=99]{biblatex}

\title{Knowledge Refinement in\\Expressive Description Logics}
\author{Roland Bernard}
\date{July 2023}

\definecolor{unibzblue}{HTML}{0086ec}
\definecolor{unibzblue2}{HTML}{f0f8ff}

\setbeamercolor{outerbox}{fg=black,bg=unibzblue2}
\setbeamercolor{innerbox}{fg=white,bg=unibzblue}

\setbeamertemplate{title page}{
  \scriptsize 
  \begin{tabular}{ll}
    \multicolumn{2}{r}{\includegraphics[width=0.3\linewidth]{resources/unibz-logo.png}} \\
    \multicolumn{2}{l}{\raggedright Bachelor Thesis} \\
    \\
    \multicolumn{2}{p{\linewidth}}{\raggedright \huge \bf \inserttitle} \\
    \vspace*{5mm} \\
    Candidate   & \insertauthor \\
    \\
    Supervisors & Oliver Kutz \\
                & Nicolas Troquard \\
    \\
    \multicolumn{2}{p{\linewidth}}{\insertdate}
  \end{tabular}
}

\setbeamertemplate{frametitle}{
  \vspace*{6mm}
  \Large \bf \color{black} \insertframetitle
}

\setbeamertemplate{itemize items}{\scriptsize\raisebox{0.6mm}{\color{unibzblue} $\blacktriangleright$}}

\newcommand{\ALC}{\ensuremath{\texorpdfstring{\mathcal{ALC}}{ALC}}\xspace}
\newcommand{\SROIQ}{\ensuremath{\texorpdfstring{\mathcal{SROIQ}}{SROIQ}}\xspace}
\newcommand{\ALCHOIQ}{\ensuremath{\texorpdfstring{\mathcal{ALCHOIQ}}{ALCHOIQ}}\xspace}
\newcommand{\ALCH}{\ensuremath{\texorpdfstring{\mathcal{ALCH}}{ALCH}}\xspace}

\newcommand{\Lmc}{\ensuremath{\mathcal{L}}\xspace}
\newcommand{\Tmc}{\ensuremath{\mathcal{T}}\xspace}
\newcommand{\Amc}{\ensuremath{\mathcal{A}}\xspace}
\newcommand{\Rmc}{\ensuremath{\mathcal{R}}\xspace}
\newcommand{\Omc}{\ensuremath{\mathcal{O}}\xspace}
\newcommand{\Omcref}{\ensuremath{{\mathcal{O}^\textnormal{ref}}}\xspace}
\newcommand{\Omcfull}{\ensuremath{{\mathcal{O}^\textnormal{full}}}\xspace}

\newcommand{\op}[1]{\operatorname{#1}}
\newcommand{\Inf}{{\ensuremath{\op{Inf}}}\xspace}
\newcommand{\sub}{{\ensuremath{\op{sub}}}\xspace}
\newcommand{\qual}{{\ensuremath{\op{IIC}}}\xspace}
\newcommand{\UpC}{{\ensuremath{\op{UpCover}}}\xspace}
\newcommand{\DownC}{{\ensuremath{\op{DownCover}}}\xspace}
\newcommand{\inv}{\ensuremath{\op{inv}}\xspace}
\newcommand{\refine}{\ensuremath{\op{\uparrow}}\xspace}
\newcommand{\corefine}{\ensuremath{\op{\downarrow}}\xspace}

\newcommand{\disjoint}{\ensuremath{\op{disjoint}}\xspace}
\newcommand{\self}{\ensuremath{\mathit{Self}}\xspace}
\newcommand{\less}[2]{\ensuremath{\leq #1~#2}\xspace}
\newcommand{\more}[2]{\ensuremath{\geq #1~#2}\xspace}
\newcommand{\nominal}[1]{\ensuremath{\{#1\}}\xspace}

\newcommand{\litem}{{\scriptsize\raisebox{0.6mm}{\color{unibzblue} $\blacktriangleright$}}}
\newcommand{\sitem}{{\scriptsize\raisebox{0.2mm}{\color{unibzblue} $\rightarrow$}}\enspace}
\newcommand{\icon}[1]{\textcolor{unibzblue}{\faIcon[solid]{#1}}}

\newlength{\offsetpage}
\newenvironment{narrowpage}[1][3mm]{
  \setlength{\offsetpage}{#1}
  \begin{adjustwidth}{\offsetpage}{-\offsetpage}
}{\end{adjustwidth}}

\newenvironment{items}{\begin{narrowpage}}{\end{narrowpage}}
\newenvironment{oitem}[2][3mm]{\hfill\begin{beamercolorbox}[rounded=false,colsep=#1]{outerbox}\parbox[t]{4mm}{\centering#2}\parbox[t]{4mm}{\ }\begin{minipage}[t]{\textwidth}}{\end{minipage}\end{beamercolorbox}\hfill\vspace{2mm}}

\beamertemplatenavigationsymbolsempty
\setbeamertemplate{footline}{}

\begin{document}

\frame{\titlepage}

\begin{frame}
  \frametitle{Description Logics}
  \begin{items}
    \begin{oitem}{\litem}
      Family of logics used to represent knowledge \\
      \sitem {\footnotesize aiming for favorable trade-offs between complexity and expressivity}
    \end{oitem}
    \begin{oitem}{\litem}
      \begin{tabular}[t]{@{}ll}
        {\footnotesize Individuals,} & {\footnotesize  e.g.,} \enspace $\mathit{roland}$ \quad $\mathit{oliver}$ \quad $\mathit{nicolas}$ \quad $\mathit{unibz}$ \quad $\mathit{blue}$ \\
        {\footnotesize Concepts,} & {\footnotesize  e.g.,} \enspace $\mathit{Student}$ \quad $\mathit{Professor}$ \quad $\mathit{Person}$ \quad $\mathit{University}$ \quad $\mathit{Color}$ \\
        {\footnotesize Roles,} & {\footnotesize  e.g.,} \enspace $\mathit{studiesAt}$ \quad $\mathit{supervisedBy}$ \quad $\mathit{hasColor}$ \\
      \end{tabular}
    \end{oitem}
    \begin{oitem}{\litem}
      \begin{tabular}[t]{@{}ll}
        {\footnotesize Complex concepts,} & {\footnotesize  e.g.,} \enspace $\lnot \mathit{Person}$ \quad $\mathit{Person} \sqcap \mathit{Student}$ \quad $\exists studiesAt.University$
      \end{tabular}
    \end{oitem}
    \begin{oitem}{\litem}
      \begin{tabular}[t]{@{}ll}
        {\footnotesize Axioms,} & {\footnotesize  e.g.,} \enspace $\mathit{Student} \sqsubseteq \mathit{Person}$ \quad $\mathit{Student}(\mathit{roland})$ \quad $\mathit{studiesAt}(roland, unibz)$
      \end{tabular}
    \end{oitem}
  \end{items}
\end{frame}

\begin{frame}
  \frametitle{Knowledge Refinement in Description Logics}
  \begin{items}
    \begin{oitem}{\litem}
      Process of iteratively modifying and improving the ontology
    \end{oitem}
    \begin{oitem}{\litem}
      Using two refinement operators \\
      \begin{tabular}[t]{@{}ll}
        \sitem {\footnotesize specialization operator,} & {\footnotesize  e.g.,} \enspace $\mathit{Student}$ {\footnotesize is a specialization of} $\mathit{Person}$ \\
        \sitem {\footnotesize generalization operator,} & {\footnotesize  e.g.,} \enspace $\mathit{Person}$ {\footnotesize is a generalization of} $\mathit{Student}$ \\
      \end{tabular}
    \end{oitem}
    \begin{oitem}{\litem}
      Using an axiom weakening operator \\
      \sitem {\footnotesize uses the two refinement operators} \\
      \sitem {\footnotesize generates axioms that are less restrictive}
    \end{oitem}
  \end{items}
\end{frame}

\begin{frame}
  \frametitle{Applications of Knowledge Refinement}
  \begin{items}
    \begin{oitem}{\icon{tools}}
      Repairing ontologies, e.g., \\
      \sitem {\footnotesize making inconsistent ontologies consistent} \\
      \sitem {\footnotesize removing unintended consequences}
    \end{oitem}
    \begin{oitem}{\icon{balance-scale}}
      Combination of conflicting knowledge \\
      \sitem {\footnotesize also for computational concept combination}
    \end{oitem}
    \begin{oitem}{\icon{robot}}
      Machine learning \\
      \sitem {\footnotesize learning axioms from data}
    \end{oitem}
  \end{items}
\end{frame}

\begin{frame}
  \frametitle{Weakening in Expressive Description Logics}
  \begin{items}
    \begin{oitem}{\icon{bullseye}}
      Extend axiom weakening to the description logic \SROIQ
    \end{oitem}
    \begin{oitem}{\icon{exclamation}}
      Only simple roles can be used in every context \\
      \sitem {\footnotesize using a non-simple role in some places is forbidden}
    \end{oitem}
    \begin{oitem}{\icon{exclamation}}
      The graph formed by role inclusions must match some conditions \\
      \sitem {\footnotesize adding new role inclusions can cause violations}
    \end{oitem}
    \begin{oitem}{\icon{shield-alt}}
      These problems have been prevented by ensuring that \\
      \sitem {\footnotesize all simple roles remain simple after refinement} \\
      \sitem {\footnotesize only simple roles are used during the refinement}
    \end{oitem}
  \end{items}
\end{frame}

\begin{frame}
  \frametitle{A Protégé Plugin supporting Axiom Weakening}
  \begin{items}
    \begin{oitem}{\icon{puzzle-piece}}
      Protégé plugin for axiom weakening \\
      \sitem {\footnotesize allow computing weakening for specific axioms} \\
      \sitem {\footnotesize enable automatic ontology repair}
    \end{oitem}
    \includegraphics[width=\textwidth]{resources/protege-both.png}
  \end{items}
\end{frame}

\begin{frame}
  \frametitle{Evaluating Axiom Weakening for Ontology Repair}
  \begin{items}
    \begin{oitem}{\icon{wrench}}
      Repaired once using axiom weakening and once using removal  \\
      \sitem {\footnotesize the quality of the resulting repairs is compared}
    \end{oitem}
    \begin{oitem}{\icon{question-circle}}
      Deciding which repair is ``better'' is not well-defined \\
      \sitem {\footnotesize we want to retain as many consequences as possible} \\
      \sitem {\footnotesize we focus only on subsumption between simple concepts}
    \end{oitem}
    \begin{oitem}{\icon{balance-scale-left}}
      For comparing two repairs we define the IIC of $\Omc_1$ w.r.t $\Omc_2$ \\
      \sitem {\footnotesize value close to $1$ for when $\Omc_1$ is better} \\
      \sitem {\footnotesize $0.5$ if both repairs are equally ``good''} \\
      \sitem {\footnotesize value close to $0$ for when $\Omc_2$ is better}
    \end{oitem}
  \end{items}
\end{frame}

\begin{frame}
  \frametitle{Evaluation Results}
  \begin{items}
    \begin{oitem}{\icon{chart-bar}}
      Comparison between using axiom weakening and using removal \\
      \sitem {\footnotesize significantly better for some ontologies} \\
      \sitem {\footnotesize in many cases only minor or no improvement}
    \end{oitem}
    \includegraphics[width=\textwidth]{resources/iic-remove-ontology-bar.png}
  \end{items}
\end{frame}

\begin{frame}
  \frametitle{Outcomes of the Thesis}
  \begin{items}
    \begin{oitem}{\icon{plus}}
      Extended the axiom weakening operator to \SROIQ \\
      \sitem {\footnotesize and showed that the proposed approach maintains the necessary constraints}
    \end{oitem}
    \begin{oitem}{\icon{puzzle-piece}}
      Developed a Protégé plugin for applying these techniques \\
      \sitem {\footnotesize allowing users to easily repair ontologies and weaken axioms}
    \end{oitem}
    \begin{oitem}{\icon{chart-bar}}
      Evaluated the proposed approach on real-world ontologies \\
      \sitem {\footnotesize showing that axiom weakening can outperform removal}
    \end{oitem}
  \end{items}
\end{frame}

{
\setbeamercolor{background canvas}{bg=black}
\begin{frame}[plain]{}
\end{frame}
}

\begin{frame}
  \frametitle{Expressive \textnormal{Description Logics}}
  \begin{items}
    \begin{oitem}{\litem}
      Additional kinds of concept expressions and axioms, e.g., \\
      \begin{tabular}[t]{@{}ll}
        \sitem {\footnotesize role inclusions,} &  {\footnotesize  e.g.,} \enspace $\mathit{motherOf} \sqsubseteq \mathit{parentOf}$ \\
        \sitem {\footnotesize cardinality constraints,} &  {\footnotesize  e.g.,} \enspace $\less 1 {\mathit{studiesAt}.\mathit{University}}$ \quad $\more 2 {\mathit{supervisedBy}.\top}$ \\
      \end{tabular}
    \end{oitem}
    \begin{oitem}{\litem}
      Require additional rules to guarantee decidability \\
      \sitem {\footnotesize separation into simple and non-simple roles} \\
      \sitem {\footnotesize limits on the graph formed by role inclusions}
    \end{oitem}
    \begin{oitem}{\icon{bullseye}}
      Focus on \SROIQ and the Web Ontology Language
    \end{oitem}
  \end{items}
\end{frame}

\begin{frame}
  \frametitle{Implementation of Axiom Weakening}
  \begin{items}
    \begin{oitem}{\icon{java}}
      Implemented in Java using the OWL API \\
      \sitem {\footnotesize a library providing a uniform way of interfacing with reasoners}
    \end{oitem}
    \begin{oitem}{\litem}
      Using of-the-shelf reasoners for the Web Ontology Language \\
      \sitem {\footnotesize requires mapping between Web Ontology Language and \SROIQ}
    \end{oitem}
    \begin{oitem}{\icon{check-circle}}
      Tests to ensure correct behavior of the implementation \\
      \sitem {\footnotesize manual tests of expected operator results} \\
      \sitem {\footnotesize automatically generated tests asserting general properties and invariants}
    \end{oitem}
  \end{items}
\end{frame}

\begin{frame}
  \frametitle{Future Outlook}
  \begin{items}
    \begin{oitem}{\icon{lock-open}}
      Study the possibility of loosening the restriction \\
      \sitem {\footnotesize refine with non-simple roles in some cases} \\
      \sitem {\footnotesize more permissive weakening of role inclusions}
    \end{oitem}
    \begin{oitem}{\icon{bullseye}}
      Study better ways of guiding the repair process \\
      \sitem {\footnotesize using better heuristics, maybe domain specific} \\
      \sitem {\footnotesize using user input to guide the repairs}
    \end{oitem}
    \begin{oitem}{\icon{ruler}}
      Find better measures for comparing the quality of repairs
    \end{oitem}
    \begin{oitem}{\icon{lightbulb}}
      Study other possible applications of axiom weakening
    \end{oitem}
  \end{items}
\end{frame}

\end{document}
